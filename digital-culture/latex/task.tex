\documentclass{article}
\usepackage[utf8]{inputenc}
\usepackage[T2A]{fontenc}
\usepackage[russian]{babel}
\usepackage{amsfonts}
\usepackage{amsmath}
\usepackage{amssymb}
\usepackage{arcs}
\usepackage{fancyhdr}
\usepackage{float}
\usepackage[left=3cm,right=3cm,top=3cm,bottom=3cm]{geometry}
\usepackage{graphicx}
\usepackage{hyperref}
\usepackage{multicol}
\usepackage{stackrel}
\usepackage{xcolor}

%
% @author Saveliy Bakturin
% <p>
% Don't write off, if you don't wanna be banned!
%

\begin{document}
	\pagestyle{empty}
	\normalsize
	\section{Матанализ от Виноградова}
	\subsection{}
	Пусть $k \in [1:n], r \in \mathbb{N} \cup \{\infty\}$. Множество $M \subset \mathbb{R}^n$ называется \textbf{гладким} $k$\textbf{-мерным многообразием класса} $C^{(r)}$ или $r$\textbf{-гладким} $k$\textbf{-мерным многообразием}, если для любого $x \in M$ существует окрестность $V_{x}^{M}$ и регулярный класса $C^{(r)}$ гомеоморфизм $\varphi : $ П$_{k} \rightarrow V_{x}^{M}$, такой что $\varphi(\mathbb{O}_{k})=x$.
	\subsection{}
	\textbf{Теорема (Планшерель)}. Если $f \in L_{1}(\mathbb{R}) \cap L_{2}(\mathbb{R})$, то $\widehat{f} \in L_{2}(\mathbb{R})$ и
	\[\int_{\mathbb{R}}|\widehat{f}|^{2}=\int_{\mathbb{R}}|f|^{2}.\]

	\textbf{Теорема' (Планшерель).} Классическое преобразование Фурье продолжается с множества $L_{1}(\mathbb{R}) \cap L_{2}(\mathbb{R})$ до унитарного оператора в комплексном пространстве $L_{2}(\mathbb{R})$. \\
	\textbf{Следствие}. \textbf{Обобщенная теорема Планшереля}. Если $f, g \in L_{1}(\mathbb{R}) \cap L_{2}(\mathbb{R})$, то
	\[\int_{\mathbb{R}}\widehat{f}\bar{\widehat{g}} = \int_{\mathbb{R}}f\bar{g}.\]

	\section{Большое задание от доктора Тренча}
	\subsection{}
	Since $f(0)=f'(1)=f''(0)=0$ and $f'''(x)=12(2x-1)$, Theorem 11.3.5d and Exercise 11.3.50b imply that
	\begin{align*}
		\alpha_{n} &= -\frac{192}{(2n-1)^{3}\pi^{3}}\int_{0}^{1}(2x-1)\cos{\frac{(2\pi-1)\pi x}{2}}dx \\
		&= -\frac{384}{(2n-1)^{4}\pi^{4}}\left[(2x+1)\sin{\frac{(2n-1)\pi x}{2}}\bigg|_{0}^{1}-2\int_{0}^{1}\sin{\frac{(2n-1)\pi x}{2}}dx\right] \\
		&= -\frac{384}{(2n-1)^{4}\pi^{4}}\left[(-1)^{n+1}1+\frac{4}{(2n-1)\pi}\cos{\frac{(2n-1)\pi x}{2}}\bigg|_{0}^{1}\right] \\
		&= -\frac{384}{(2n-1)^{4}\pi^{4}}\left[(-1)^{n+1}1+\frac{4}{(2n-1)\pi}\right] = \frac{384}{(2n-1)^{4}\pi^{4}}\left[(-1)^{n}+\frac{4}{(2n-1)\pi}\right];
	\end{align*}

	$\displaystyle S_{M}(x)=\frac{384}{\pi^{4}}\sum\limits_{n=1}^{\infty}\frac{1}{(2n-1)^{4}}\left[(-1)^{n}+\frac{4}{(2n-1)\pi}\right]\sin{\frac{(2n-1)\pi x}{2}}. \text{ From Definition 12.1.4},$
	$$\quad \quad \quad u(x,t)=\frac{384}{\pi^{4}}\sum\limits_{n=1}^{\infty}\frac{1}{(2n-1)^{4}}\left[(-1)^{n}+\frac{4}{(2n-1)\pi}\right]e^{-(2n-1)^{2}\pi^{2}t}\sin{\frac{(2n-1)\pi x}{2}}.$$

	\section{Маленькие задания от доктора Тренча}
	\subsection{}
	$\displaystyle t^{2} \leftrightarrow \frac{2}{s^{3}} \text{ and } t^{3} \leftrightarrow \frac{6}{s^{4}} \text{, so } H(s)=\frac{12}{s^{7}}$.
	\subsection{}
	$\displaystyle t^{7} \leftrightarrow \frac{7!}{s^{8}} \text{ and } e^{-t}\sin{2t} \leftrightarrow \frac{2}{(s+1)^{2}+4} \text{, so } H(s)=\frac{2\cdot7!}{s^{8}[(s+1)^{2}+4]}$.
	\subsection{}
	$\displaystyle \sinh{at} \leftrightarrow \frac{a}{s^{2}-a^{2}} \text{ and } \cosh{at} \leftrightarrow \frac{1}{s^{2}-a^{2}} \text{, so } H(s)=\frac{as}{(s^{2}-a^{2})^{2}}$.
\end{document}
