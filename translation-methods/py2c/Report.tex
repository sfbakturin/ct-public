\documentclass[12pt, a4paper, oneside, final]{article}
\usepackage[margin = 0.75in, bottom = 0.75in]{geometry}

\pdfminorversion = 5 
\pdfcompresslevel = 9
\pdfobjcompresslevel = 2

\usepackage[T2A]{fontenc}
\usepackage[utf8]{inputenc}
\usepackage[english, russian]{babel}
\usepackage{xcolor, ulem, soulutf8, soul, fancyhdr, amsmath, amssymb, amsthm, wrapfig, csvsimple, float, caption, subcaption, titlesec, hyperref, multicol, listings, tocloft, longtable, skak, stmaryrd, color, minted}
\usepackage[shortlabels]{enumitem}
\usepackage[most]{tcolorbox}
\usepackage[framemethod = tikz]{mdframed}

\definecolor{codegreen}{rgb}{0, 0.6, 0}
\definecolor{codegray}{rgb}{0.5, 0.5, 0.5}
\definecolor{codepurple}{rgb}{0.58, 0, 0.82}
\definecolor{backcolour}{rgb}{0.95, 0.95, 0.92}

\hypersetup{colorlinks, citecolor = pink, filecolor = pink, linkcolor = pink, urlcolor = pink}

\usemintedstyle{perldoc}
\setminted{tabsize = 4, breaklines, autogobble, frame = single, framesep = 1mm}

\everymath{\displaystyle}
\binoppenalty = 10000
\relpenalty = 10000
\sloppy

\newtcblisting{pseudocode}{
	listing only,
	breakable,
	colback = backcolour,
	enhanced jigsaw,
	sharp corners,
	boxrule = 0pt,
	frame hidden,
	listing options = {
		mathescape,
		commentstyle = \color{codegreen},
		keywordstyle = \color{magenta},
		numberstyle = \tiny\color{codegray},
		stringstyle = \color{codepurple},
		basicstyle = \ttfamily\footnotesize,
		breakatwhitespace = false,
		breaklines = true,
		captionpos = b,
		keepspaces = true,
		numbers = left,
		numbersep = 5pt,
		showspaces = false,
		showstringspaces = false,
		showtabs = false,
		tabsize = 4,
		inputencoding = utf8,
		language = python
	}
}

\renewcommand*{\theenumi}{\thesection.\arabic{enumi}}
\renewcommand*{\theenumii}{\alph{enumii}}
\renewcommand*{\labelitemi}{\ensuremath{\triangleright}}

\def\cover{
	\begin{center}
		{Национальный исследовательский университет ИТМО \\ Факультет информационных технологий и программирования \\ Прикладная математика и информатика} \\ [5.0em]
		{\Huge \bfseries Методы трансляции} \\ [0.5em]
		{\large Отчет по лабораторной работе №3} \\ [0.5em]
		\textcolor{gray}{\textlangle Собрано \today \textrangle}
	\end{center}
}

\begin{document}
	\thispagestyle{empty}
	\vspace*{0.5em}
	\cover
	\begingroup
	\vspace*{30em}
	\def\hd{\begin{tabular}{ll}
			\textbf{Работу выполнил:} \\ {Бактурин Савелий Филиппович M33331} \vspace*{1em} \\
			\textbf{Преподаватель:} \\ {Станкевич А. С.} \vspace*{1em} \\
		\end{tabular}
	}
	\newlength{\hdwidth}
	\settowidth{\hdwidth}{\hd}
	\hfill\begin{minipage}{\hdwidth}\hd\end{minipage}
	\endgroup
	\newpage
	\section{Введение}
	\textbf{Перевод с Python на Си}. Необходимо было выбрать подмножество языка Python и написать транслятор, который переводит программы на заданном подмножестве на язык Си, причем если на входе программа корректна, то и на выходе должна быть корректна.

	В выбранном подмножестве были реализованы следующие стандартные Python типы: \texttt{int}, \texttt{float}, \texttt{str}. Поддерживаются следующие конструкции: \texttt{while}, \texttt{if}, \texttt{elif}, \texttt{else}. Также поддерживается возможность создания функций, не принимающих ничего в качестве аргументов и возвращающих стандартный тип \texttt{void}:
	\begin{minted}{c}
		void func_name(void);
	\end{minted}
	\section{Грамматика}
	Для парсинга была выбрана следующая грамматика:
	\begin{itemize}
		\item Непосредственно программа:
		\begin{align*}
			\mathtt{Python} &\to (\mathtt{Statement})\texttt{*}~\mathtt{EOF}
		\end{align*}
		\item Изложение или сущность в Python:
		\begin{align*}
			\mathtt{Statement} &\to (\mathtt{AssignmentStatement}~\texttt{|}~\mathtt{CallStatement}~\texttt{|}~\mathtt{CompoundStatement})
		\end{align*}
		\item Виды сущностей:
		\begin{align*}
			\mathtt{AssignmentStatement} &\to \mathit{Identifier}~'\texttt{=}'~\mathtt{Expression} \\
			\mathtt{CallStatement} &\to \mathit{Identifier}~'\texttt{(}'~\mathtt{CallArguments}~'\texttt{)}' \\
			\mathtt{CompoundStatement} &\to (\mathtt{IfStatement}~\texttt{|}~\mathtt{WhileStatement}~\texttt{|}~\mathtt{DefStatement}~\texttt{|}~\mathtt{ForStatement})
		\end{align*}
		где $\mathtt{Expression}$~-- это верное математическое выражение с установленными приоритетами (подобные в Python), $\mathtt{CallArguments}$~-- это множество математических выражений, разделенных терминалом $'\texttt{,}'$.
		\item Составные сущности:
		\begin{align*}
			\mathtt{IfStatement} &\to '\texttt{if}'~\mathtt{Expression}~'\texttt{:}'~\mathtt{Block}~(\mathtt{ElifStatement})\texttt{*}~(\mathtt{ElseStatement})\texttt{?} \\
			\mathtt{WhileStatement} &\to '\texttt{while}'~\mathtt{Expression}~'\texttt{:}'~\mathtt{Block} \\
			\mathtt{DefStatement} &\to '\texttt{def}'~\mathit{Identifier}~'\texttt{(}'~'\texttt{)}'~'\texttt{:}'~\mathtt{Block} \\
			\mathtt{ForStatement} &\to '\texttt{for}'~\mathit{Identifier}~'\texttt{in}'~(\mathtt{Range1Statement}~\texttt{|}~\mathtt{Range2Statement})~'\texttt{:}'~\mathtt{Block}
		\end{align*}
		где $\mathtt{Block}$~-- это один и больше изложений с обязательной табуляцией, $\mathtt{Range1Statement}$ и $\mathtt{Range2Statement}$~-- это \mintinline{python}|range| с одним или двумя аргументами и $\mathtt{ElifStatement}$, $\mathtt{ElseStatement}$~-- продолжение $\mathtt{IfStatement}$.
	\end{itemize}
	\section{Технические подробности}
	Использованный язык программирования~-- \textit{Java}, выбранная система сборки~-- \textit{Maven} с описанными зависимостями.

	Подготовлен банк тестов (всего~-- 10), использующий библиотеку \textit{JUnit4} \url{https://junit.org/junit4/} для проверки строгой типизации.
	\section{Тесты}
	Здесь описаны и показаны тесты-программы на Python.
	\begin{itemize}
		\item Пустая программа является валидной с точки зрения Python.
		\inputminted{python}{./samples/empty.py}
		\item Программа, выводящая "Hello world" в стандартный поток вывода.
		\inputminted{python}{./samples/hello_world.py}
		\item Математические выражения.
		\inputminted{python}{./samples/calc.py}
		\item Работоспособность \mintinline{python}|if|-конструкции.
		\inputminted{python}{./samples/if.py}
		\item Работоспособность \mintinline{python}|def|-конструкции с функцией.
		\inputminted{python}{./samples/func.py}
		\item Работоспособность \mintinline{python}|while|-конструкции.
		\inputminted{python}{./samples/while.py}
		\item Работоспособность \mintinline{python}|for|-конструкции.
		\inputminted{python}{./samples/for.py}
	\end{itemize}
	Ниже представлены тесты, который должны закончится неудачно с точки зрения компиляции.
	\begin{itemize}
		\item Смешивание Си-типов: \mintinline{c}|int|, \mintinline{c}|float|, \mintinline{c}|char*|.
		\inputminted{python}{./samples/error_types.py}
		\item Битовые операции с целыми и нецелыми числами.
		\inputminted{python}{./samples/error_binary.py}
		\item Нереализованные возможности языка Python (например, умножение строки на число~-- повторить строку $n$ раз).
		\inputminted{python}{./samples/error_not_implemented_yet.py}
	\end{itemize}
\end{document}