\documentclass[12pt, a4paper, oneside, final]{article}
\usepackage[margin = 0.75in, bottom = 0.75in]{geometry}

\pdfminorversion = 5 
\pdfcompresslevel = 9
\pdfobjcompresslevel = 2

\usepackage[T2A]{fontenc}
\usepackage[utf8]{inputenc}
\usepackage[english, russian]{babel}
\usepackage{xcolor, ulem, soulutf8, soul, fancyhdr, amsmath, amssymb, amsthm, wrapfig, csvsimple, float, caption, subcaption, titlesec, hyperref, multicol, listings, tocloft, longtable, skak, stmaryrd, color, minted}
\usepackage[shortlabels]{enumitem}
\usepackage[most]{tcolorbox}
\usepackage[framemethod = tikz]{mdframed}

\definecolor{codegreen}{rgb}{0, 0.6, 0}
\definecolor{codegray}{rgb}{0.5, 0.5, 0.5}
\definecolor{codepurple}{rgb}{0.58, 0, 0.82}
\definecolor{backcolour}{rgb}{0.95, 0.95, 0.92}

\hypersetup{colorlinks, citecolor = pink, filecolor = pink, linkcolor = pink, urlcolor = pink}

\setminted{tabsize = 2, breaklines, autogobble, fontsize = \footnotesize}

\newenvironment{code}{\captionsetup{type=listing}}{}

\everymath{\displaystyle}
\binoppenalty = 10000
\relpenalty = 10000
\sloppy

\newtcblisting{pseudocode}{
	listing only,
	breakable,
	colback = backcolour,
	enhanced jigsaw,
	sharp corners,
	boxrule = 0pt,
	frame hidden,
	listing options = {
		mathescape,
		commentstyle = \color{codegreen},
		keywordstyle = \color{magenta},
		numberstyle = \tiny\color{codegray},
		stringstyle = \color{codepurple},
		basicstyle = \ttfamily\footnotesize,
		breakatwhitespace = false,
		breaklines = true,
		captionpos = b,
		keepspaces = true,
		numbers = left,
		numbersep = 5pt,
		showspaces = false,
		showstringspaces = false,
		showtabs = false,
		tabsize = 4,
		inputencoding = utf8,
		language = python
	}
}

\renewcommand*{\theenumi}{\thesection.\arabic{enumi}}
\renewcommand*{\theenumii}{\alph{enumii}}
\renewcommand*{\labelitemi}{\ensuremath{\triangleright}}

\def\cover{
	\begin{center}
		{Национальный исследовательский университет ИТМО \\ Факультет информационных технологий и программирования \\ Прикладная математика и информатика} \\ [5.0em]
		{\Huge \bfseries Методы трансляции} \\ [0.5em]
		{\large Отчет по лабораторной работе №2} \\ [0.5em]
		\textcolor{gray}{\textlangle Собрано \today \textrangle}
	\end{center}
}

\begin{document}
	\thispagestyle{empty}
	\vspace*{0.5em}
	\cover
	\begingroup
	\vspace*{30em}
	\def\hd{\begin{tabular}{ll}
			\textbf{Работу выполнил:} \\ {Бактурин Савелий Филиппович M33331} \vspace*{1em} \\
			\textbf{Преподаватель:}   \\ {Станкевич А. С.} \vspace*{1em} \\
		\end{tabular}
	}
	\newlength{\hdwidth}
	\settowidth{\hdwidth}{\hd}
	\hfill\begin{minipage}{\hdwidth}\hd\end{minipage}
	\endgroup
	\newpage
	\section{Грамматика}
	\textbf{Описания переменных в Си}. Сначала следует имя типа, затем разделенные запятой имена переменных. Переменная может быть указателем, в этом случае перед ней идет звездочка (возможны и указатели на указатели, и так далее). Описаний может быть несколько.
	
	Предлагаемая КС-грамматика языка описания переменных в Си:
	\begin{align*}
		S & \to T + \texttt{'\ '} + V + \texttt{';'}                          \\
		V & \to \texttt{'*'} + V                                              \\
		V & \to \texttt{'[a-zA-Z0-9\_]+'} + \texttt{','} + V                  \\
		V & \to \texttt{'[a-zA-Z0-9\_]+'}                                     \\
		T & \to \texttt{'struct'} + \texttt{'\ '} + \texttt{'[a-zA-Z0-9\_]+'} \\
		T & \to \texttt{<keyword>} + \texttt{'\ '} + K                        \\
		T & \to \texttt{'[a-zA-Z0-9\_]+'}                                     \\
		K & \to \texttt{<keyword>} + \texttt{'\ '} + K                        \\
		K & \to \varepsilon
	\end{align*}
	
	Множество \texttt{<keyword>} состоит из ключевых слов языка Си \url{https://en.cppreference.com/w/c/keyword}, допустимых при описании оных переменных. К ним могут относиться модификаторы: \texttt{const}, \texttt{volatile}, \texttt{static}, а также стандартные арифметические типы \url{https://en.cppreference.com/w/c/language/arithmetic_types}.
	\section{Технические подробности}
	Использованный язык программирования~-- \textit{Java}, выбранная система сборки~-- \textit{Maven} с описанными зависимостями.
	
	Для получения графической интерпретации разбора выражения используется система \textit{GraphViz} \url{https://graphviz.org/} (содержимое \texttt{.dot} файл~-- \mintinline{java}|CVarDefTree.getDot()|), для запуска используйте скрипт \texttt{Image.sh}.
	
	Проверка валидности комбинации стандартных арифметических типов осуществляется в структуре дерева разбора при вызове метода \mintinline{java}|CVarDefTree.getVariables()|.
	
	Подготовлен банк тестов (всего~-- 34), использующая библиотеку \textit{JUnit4} \url{https://junit.org/junit4/} для общей проверки работоспособности приложения.
	\section{Исходный код}
	\begin{code}
		\inputminted{java}{./src/main/java/bakturin/lab2/Main.java}
		\caption{Интерактивное приложение}
	\end{code}
	\begin{code}
		\inputminted{java}{./src/main/java/bakturin/lab2/Graphics.java}
		\caption{DOT-вывод}
	\end{code}
	\begin{code}
		\inputminted{java}{./src/main/java/bakturin/lab2/cvardef/CVarDef.java}
		\caption{Класс, объединяющий всё для парсинга}
	\end{code}
	\begin{code}
		\inputminted{java}{./src/main/java/bakturin/lab2/cvardef/CVarDefVariable.java}
		\caption{Класс-обертка после парсинга переменных}
	\end{code}
	\begin{code}
		\inputminted{java}{./src/main/java/bakturin/lab2/cvardef/assets/CVarDefLexer.java}
		\caption{Лексический анализатор}
	\end{code}
	\begin{code}
		\inputminted{java}{./src/main/java/bakturin/lab2/cvardef/assets/CVarDefParser.java}
		\caption{Синтаксический анализатор}
	\end{code}
	\begin{code}
		\inputminted{java}{./src/main/java/bakturin/lab2/cvardef/assets/CVarDefTree.java}
		\caption{Дерево разбора}
	\end{code}
	\begin{code}
		\inputminted{java}{./src/test/java/CVarDefTester.java}
		\caption{Дополнительный компонент для тестирования}
	\end{code}
	\begin{code}
		\inputminted{java}{./src/test/java/CVarDefTests.java}
		\caption{Тесты}
	\end{code}
\end{document}
